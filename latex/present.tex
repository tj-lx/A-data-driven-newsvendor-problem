\documentclass[a4paper,12pt]{article}

% 宏包设置
\usepackage{amsmath, amssymb, amsfonts} % 数学公式支持
\usepackage{geometry} % 页面布局
\usepackage{graphicx} % 图片插入
\usepackage{booktabs} % 表格美化
\usepackage{hyperref} % 超链接
\usepackage{enumitem} % 列表格式
\usepackage{float}
\usepackage[UTF8]{ctex}

% 页面边距
\geometry{left=2.5cm,right=2.5cm,top=2.5cm,bottom=2.5cm}

% 标题信息
\title{\textbf{From Data to Decision: A Data-Driven Approach to the Newsvendor Problem}\\
\large —— 课程文献阅读与复现报告}
\author{汇报人:[您的姓名] \\ 课程:高级应用统计 (Advanced Applied Statistics)}
\date{\today}

\begin{document}

\maketitle

\begin{abstract}
本文深度解读了 Huber 等人发表于 \textit{EJOR} (2019) 的论文。文章针对报童问题中需求分布未知的核心挑战,提出了基于大数据的三层决策框架。我们详细阐述了从需求预测到库存优化的统计方法论,并基于 Kaggle 的 French Bakery 数据集进行了实证复现,验证了非参数方法在库存决策中的有效性。
\end{abstract}

% 第一部分:基本研究问题
\section{The Basic Research Problem}

\subsection{商业背景与权衡}
本文聚焦于零售管理中经典的\textbf{报童问题 (Newsvendor Problem)}。零售商(如连锁面包店)需在销售季节前决定易腐产品的订货量 $q$。核心权衡在于最小化期望总成本:
\begin{equation}
    \min_{q} \mathbb{E}[C(q, D)] = \mathbb{E} [c_u (D - q)^+ + c_o (q - D)^+]
\end{equation}
其中 $D$ 为随机需求,$c_u$ 为缺货成本 (Underage cost),$c_o$ 为超储成本 (Overage cost)。

\subsection{统计学挑战}
在理论最优解中,订货量 $q^*$ 取决于需求累积分布函数 $F$ 的分位数:$q^* = F^{-1}(\frac{c_u}{c_u + c_o})$。
然而,现实中的\textbf{根本难题}在于:
\begin{itemize}
    \item \textbf{分布未知 (Unknown Distribution)}:真实的需求分布 $F$ 往往无法获知,且可能随时间变化。
    \item \textbf{特征利用不足}:传统方法往往忽略了天气、节假日、促销等外部特征向量 $X$ 对需求分布的影响。
\end{itemize}
因此,本文的研究问题是:\textbf{如何利用历史数据 $(D_t, X_t)$,在不预设分布形式的前提下,构建数据驱动的模型以实现成本最小化?}

% 第二部分:思想与方法论
\section{The Idea and Methodology}

本文提出了一个\textbf{三层递进的数据驱动框架},涵盖了从参数估计到非参数优化的完整路径。

\subsection{Level 1: Demand Estimation (点预测)}
利用统计学习方法估计给定特征 $x$ 下的需求期望 $\hat{y}(x) = \mathbb{E}[d|x]$。
\begin{itemize}
    \item \textbf{传统方法}:ARIMA, ETS (指数平滑)。
    \item \textbf{机器学习 (ANN)}:文章构建了单隐层多层感知机 (MLP)。
    \begin{equation}
        \hat{y}(x) = f(W^{(2)} \sigma(W^{(1)}x + b^{(1)}) + b^{(2)})
    \end{equation}
    通过引入特征工程(如滞后销量 $Lag\_1, Lag\_7$ 和日历特征),捕捉非线性模式。
\end{itemize}

\subsection{Level 2: Inventory Optimization (库存优化)}
基于预测结果 $\hat{y}$ 和预测误差 $\epsilon = d - \hat{y}$ 进行决策。
\begin{itemize}
    \item \textbf{Parametric (Model-based)}:假设误差服从正态分布 $\epsilon \sim \mathcal{N}(0, \hat{\sigma}^2)$。
    $$q(x) = \hat{y}(x) + \Phi^{-1}(\text{target ratio}) \cdot \hat{\sigma}$$
    \item \textbf{Non-parametric (SAA)}:\textbf{样本均值逼近 (Sample Average Approximation)}。不假设分布,直接使用历史误差样本的经验分布寻找分位数。
    \textbf{优势}:避免了模型误设 (Misspecification) 风险,更具鲁棒性。
\end{itemize}

\subsection{Level 3: Integrated Estimation (集成优化)}
跳过点预测,直接建立特征 $x$ 到最优订货量 $q^*$ 的映射。这等价于\textbf{分位数回归 (Quantile Regression)}。
我们将报童损失函数直接作为神经网络的训练目标 (Loss Function):
\begin{equation}
    \mathcal{L} = \frac{1}{n} \sum_{i=1}^{n} \left[ c_u (d_i - q(x_i))^+ + c_o (q(x_i) - d_i)^+ \right]
\end{equation}
该方法能自动适应\textbf{异方差性}(即需求的波动随特征变化)。

% 第三部分:结果(包含论文结果与复现结果)
\section{Data and Preparation}
本研究基于 \textbf{French Bakery Daily Sales} 数据,原始为交易级记录。为复现论文方法,我们进行了数据重构与清洗:
\begin{itemize}
    \item 聚合到\textbf{日粒度}:按 \texttt{date, article} 汇总销量与收入,并对缺失日期进行\textbf{补零},保留零销量日。
    \item 价格解析与约束:解析含欧元符号与逗号小数的单价,计算 \texttt{avg\_price};当 \texttt{sales>0} 且 \texttt{avg\_price} 缺失或非正数时\textbf{删除该行}。
    \item 异常过滤:依据零销量比与非零天数剔除信息稀缺的产品,同时\textbf{保留销量 Top 10} 提升代表性。
    \item 特征工程:构造 \texttt{weekday, month, is\_public\_holiday, lag\_1, lag\_7};按时间 \textbf{80/20} 划分训练与测试集。
\end{itemize}

\section{The Results}

\subsection{论文原实证结果 (Key Findings from Paper)}
基于德国大型面包连锁店的数据,论文得出以下结论:
\begin{enumerate}
    \item \textbf{预测精度}:机器学习模型(特别是利用跨序列信息的 ANN)在 RMSE 指标上显著优于传统时间序列模型。
    \item \textbf{成本表现}:
    \begin{itemize}
        \item \textbf{SAA vs Normal}:在大部分服务水平下,数据驱动的 SAA 方法成本低于正态分布假设,证明了非参数方法的价值。
        \item \textbf{Integrated vs Separate}:集成方法在低服务水平下表现优异,但在高服务水平下对数据量要求极高,容易过拟合。
    \end{itemize}
\end{enumerate}

\subsection{小组复现结果 (Our Replication Study)}
我们基于 \textbf{Kaggle "French Bakery Daily Sales"} 数据集(Top 10 产品,约 6300 条样本)进行了完整复现。

\begin{itemize}
    \item \textbf{Level 1 预测表现}:
    \begin{itemize}
        \item \textbf{模型拟合}:Random Forest 取得了 \textbf{92.89\%} 的 $R^2$,显著优于 Linear Regression 的 90.10\%。RMSE 降低了约 15\%,验证了非线性模型在捕捉周期性需求上的优势。
    \end{itemize}
    
    \item \textbf{统计检验}:
    \begin{itemize}
        \item \textbf{Shapiro-Wilk Test}:预测残差的 p-value 为 $1.94 \times 10^{-70}$,显著拒绝正态分布假设。这为采用非参数方法 (SAA) 提供了坚实的统计学基础。
    \end{itemize}

    \item \textbf{Level 2 决策敏感性分析 (关键发现)}:
    我们测试了不同目标服务水平 (SL) 下的平均成本表现,观察到了显著的\textbf{“尾部效应” (Tail Effect)}:
    \begin{itemize}
        \item \textbf{在中低服务水平 ($SL \le 0.7$)}:Parametric (Normal) 方法表现稳健,成本略低于 SAA。这表明在分布中心区域,正态近似依然有效。
        \item \textbf{在高服务水平 ($SL \ge 0.8$)}:数据驱动的 SAA 方法开始显著跑赢。特别是在 $SL=0.95$ 时,SAA 的平均成本比 Normal 假设低约 \textbf{10\%}。
        \item \textbf{结论}:这复现并深化了原论文的观点——非参数方法 (SAA) 的核心价值在于处理\textbf{尾部风险 (Tail Risk)}。当零售商追求高服务水平(即极少缺货)时,依赖正态假设会带来巨大的潜在损失,而 SAA 展现了极强的鲁棒性。
    \end{itemize}
    
    \item \textbf{Level 3 集成优化(分位数回归)}:
    \begin{itemize}
        \item \textbf{模型与方法}:在服务水平 $\alpha=\frac{c_u}{c_u+c_o}$ 下,直接以分位损失拟合最优订货量 $q(x)$;实现采用 \texttt{GradientBoostingRegressor}(\texttt{loss=quantile}, \texttt{alpha=SL}),并对预测进行\textbf{非负裁剪}以符合业务约束。
        \item \textbf{调参与验证}:使用\textbf{时间序列交叉验证}(3 折)在网格 \texttt{n\_estimators} $\in \{100,200,400\}$、\texttt{max\_depth} $\in \{2,3,4\}$、\texttt{learning\_rate} $\in \{0.03,0.05,0.1\}$ 上搜索最优组合。
        \item \textbf{代表性结果}:例如在 $SL=0.70$ 时,最优参数为 \texttt{(200, 3, 0.05)},测试集平均成本约 \textbf{9.5468};整体上在\textbf{高服务水平}下 Integrated 未稳定优于 SAA(见后文综合曲线),提示需进一步扩展外生变量与调优网格。
    \end{itemize}
\end{itemize}

\paragraph{Level 2 决策可视化示例}
选取某一产品在一段时间内展示\textbf{真实需求}、\textbf{ML 预测}与 \textbf{SAA 订货量}。绿色与红色区域分别表示\textbf{超储}与\textbf{缺货}成本,有助于直观理解 Level 2 的库存决策效果。
\begin{figure}[H]
    \centering
    \includegraphics[width=\textwidth]{../output/inventory_result.png}
    \caption{Level 2 决策可视化:真实需求、预测与 SAA 订货量}
\end{figure}

\paragraph{Level 3 简要说明与综合成本曲线}
Level 3 采用\textbf{分位数回归}直接学习订货量 $q(x)$。下图展示 \textbf{Normal/SAA/Integrated} 在不同服务水平下的平均成本对比,其中\textbf{绿色曲线}为 Level 3 的 Integrated。结合当前特征与数据规模,Integrated 在高服务水平未稳定优于 SAA,提示需进一步扩展外生变量与调优网格。
\begin{figure}[H]
    \centering
    \includegraphics[width=\textwidth]{../output/level23_cost_curve.png}
    \caption{Level 2/3 综合:多服务水平平均成本对比(Normal/SAA/Integrated)}
\end{figure}

% 第四部分:理解与思考
\section{Understanding, Comments and Thinking}

\subsection{统计学视角的洞察}
\begin{itemize}
    \item \textbf{Prediction $\neq$ Decision}:统计上的“高预测精度”(低 MSE)并不完全等同于商业上的“低成本”。针对特定损失函数进行优化(Integrated Approach)是应用统计的高级方向。
    \item \textbf{非参数的胜利}:本研究再次印证了在“大数据”时代,基于经验分布的 SAA 方法往往比强依赖假设的参数模型更安全、更有效。
\end{itemize}

\subsection{局限与改进}
\begin{itemize}
    \item \textbf{删失数据 (Censored Data)}:当前的复现假设 $Sales \approx Demand$,忽略了缺货导致的截断。未来可引入 \textbf{Tobit 模型} 或 Survival Analysis 中的 Kaplan-Meier 估计来还原真实需求。
    \item \textbf{算法扩展}:考虑到实际数据量可能有限,未来可对比 \textbf{Random Forest} 或 \textbf{XGBoost},这些树模型通常在表格数据上比简单的 ANN 表现更稳健。
\end{itemize}

\end{document}
